\documentclass[review]{elsarticle}

\usepackage{lineno,hyperref}
\modulolinenumbers[5]

\journal{Journal of \LaTeX\ Templates}

%%%%%%%%%%%%%%%%%%%%%%%
%% Elsevier bibliography styles
%%%%%%%%%%%%%%%%%%%%%%%
%% To change the style, put a % in front of the second line of the current style and
%% remove the % from the second line of the style you would like to use.
%%%%%%%%%%%%%%%%%%%%%%%

%% Numbered
%\bibliographystyle{model1-num-names}

%% Numbered without titles
%\bibliographystyle{model1a-num-names}

%% Harvard
%\bibliographystyle{model2-names.bst}\biboptions{authoryear}

%% Vancouver numbered
%\usepackage{numcompress}\bibliographystyle{model3-num-names}

%% Vancouver name/year
%\usepackage{numcompress}\bibliographystyle{model4-names}\biboptions{authoryear}

%% APA style
%\bibliographystyle{model5-names}\biboptions{authoryear}

%% AMA style
%\usepackage{numcompress}\bibliographystyle{model6-num-names}

%% `Elsevier LaTeX' style
\bibliographystyle{elsarticle-num}
%%%%%%%%%%%%%%%%%%%%%%%

\begin{document}

\begin{frontmatter}

\title{Elsevier \LaTeX\ template\tnoteref{mytitlenote}}
\tnotetext[mytitlenote]{Fully documented templates are available in the elsarticle package on \href{http://www.ctan.org/tex-archive/macros/latex/contrib/elsarticle}{CTAN}.}

%% Group authors per affiliation:
\author{Elsevier\fnref{myfootnote}}
\address{Radarweg 29, Amsterdam}
\fntext[myfootnote]{Since 1880.}

%% or include affiliations in footnotes:
\author[mymainaddress,mysecondaryaddress]{Elsevier Inc}
\ead[url]{www.elsevier.com}

\author[mysecondaryaddress]{Global Customer Service\corref{mycorrespondingauthor}}
\cortext[mycorrespondingauthor]{Corresponding author}
\ead{support@elsevier.com}

\address[mymainaddress]{1600 John F Kennedy Boulevard, Philadelphia}
\address[mysecondaryaddress]{360 Park Avenue South, New York}

\begin{abstract}
This template helps you to create a properly formatted \LaTeX\ manuscript.
\end{abstract}

\begin{keyword}
\texttt{elsarticle.cls}\sep \LaTeX\sep Elsevier \sep template
\MSC[2010] 00-01\sep  99-00
\end{keyword}

\end{frontmatter}

\linenumbers

\section{Introduction}

Since the dawn of internet and world wide web, humanity has witnessed a degree of connection beyond reckoning. The proliferation of digital devices pervaded with various applications that account for almost all aspect of humanity, have created cyber communities that constantly mutate \cite{AtaeiACIS}; \cite{AtaeiBigDataEnvirons}. In a world where we have network infrastructures that can support up to 250Mbps of data transmission, and smart phones and IOT devices that can have processing power of up to 3 Ghz, data becomes ubiquitous, the quantum that lays the foundation of the nexus \cite{AtaeiApsec}. 

According to InternetLiveStates.com \cite{internet2019internet}, only in one second, there are 9,878 tweets sent, 1,138 instagram photos uploaded, 3,117,720 emails sent, 99,738 Google searches made, and 94,144 Youtube videos viewed. That is, if it has taken 5 second the read the preceding paragraph, during that time, 15,588,600 emails are sent. 

Driven by the ambition to harness the power of this deluge of data, the term 'Big Data' (BD) was coined \cite{lycett2013datafication}. BD initially emerged to address the challenges associated with various characteristics of data such as velocity, variety, volume and variability \cite{AtaeiBigDataEnvirons}. BD is the practice of extracting patterns, theories, and predictions from a large set of structured, semi-structured, and unstructured data for the purposes of business competitive advantage \cite{AtaeiHype}; \cite{Huberty}. BD is a game-changing innovation, heralding the dawn of a new data-oriented industry. 

Nonetheless, BD is not a magical wand that can enchant any business process. While a lot of opportunities exist in BD, subsuming an emergent and rather high-impacting technology like BD to current state of affairs in organizations, is a daunting task. According to recent survey from Databricks, only 13\% of the organizations excel at delivering on their data strategy \cite{DataBricksSurvey}. Another survey by NewVantage Partners indicated that only 24\% organization have successfully gone data-driven \cite{NewVantageSurvey}. This survey also states that only 30\% of organizations have a well established strategy for their big data endeavour. In addition, surveys from McKinsey \& Company (\cite{analytics2016age}) and Gartner (\cite{Nash}) further support these numbers, which illuminates on the scarcity of successful big data implementations in the industry.

Among the challenges of data adoption perhaps the most highlighted are 'data engineering complexities', 'big data architecture', 'rapid technology change', 'lack of sufficient skilled data engineers', and 'organization's cultural challenges of becoming data-driven' \cite{AtaeiBigDataEnvirons};\cite{Singh}. This focus of this study is on data engineering complexities and in specific big data architecture.

In the past, organization relied on a few technology giants to provide infrastructure and tools necessary for big data, while today there's a plethora of choice from hundreds of providers covering different aspect of data ecosystem from ingestion, to logging, to stream processing, and to visualization \cite{NewVantageSurvey}. Companies are tending more and more towards Cloud-native architectures for cost reduction, improved efficiency and new roles have been introduced such as chief analytics officer (CAOs) amd chief data officers (CDOs) to channel the organizational big data capabilities toward business value and competitive advantage \cite{rad2017evaluating}. 

So how can one embark on this rather sophisticated journey? what can be a good logical approach to absorb the ever-increasing complexity of big data systems? how can organizations build different stacks to handle data for various workloads such as machine learning (ML), business analytics, data engineering, and streaming? 

We suggest that majority of the challenge discussed starts with data architecture \cite{AtaeiACIS}; \cite{AtaeiApsec}. The data ingestion, processing and consumption of different data workloads vary, and sometimes they don't go well together. A company that enacted a data lake and a data warehouse and tries to account for both ecosystems, can be dealing with immense complexity, which in turns impact data teams, which in turn can hinder innovation, create barriers and result in monumental  lost.

Development and deployment of an efficacious big data system is only the beginning of a big data journey. As data sources increase, variety of data increases, number of data consumers increase, the data store gets confuscated, and this can introduce threats for scalability and maintainability of the system. This also implies that only a handful of hyper-specialized data engineers would understand the system internals, creating silos, and potential miscommunication. 

Majority of these systems are developed on-premise as ad-hoc complicated solutions that do not adhere to the practices of software engineering and software architecture \cite{Gorton}; \cite{Nadal}. As the ecosystem grows and new technologies and data processing techniques are introduced, the software architect will have a harder time to come up with a solution that address the problem requirements. 

This can potentially create grounds for an immature architecture that results in solutions that are hard to scale, hard to maintain, and raise high-entry blockades \cite{AtaeiApsec}. Since the approach of ad-hoc design to big data system development is not desirable and may leave many architects and data engineers in the dark, novel data architectures that are designed specifically for BD are required. To contribute to this goal, we explore the notion of reference architectures (RAs) and present a distributed domain-driven software RA for big data systems.


\section{Why reference architecture?}

To justify why we have chosen reference architectures as the suitable artefact, first we have to clarify two assumptions; 

\begin{enumerate}
    \item having a sound software architecture is essential to the successful development and maintenance of software systems
    \item there exist a sufficient body of knowledge in the field of software architecture to support the development of an effective RA 
\end{enumerate}

One of the focal tenets of software architecture is that every system is developed to satisfy a business objective, and that the architecture of the system is a bridge between abstract business goals to concrete final solutions \cite{SoftwareArchitectureKazman}. While the journey of big data can be quite challenging, the good news is that a software RA can be designed, analyzed and documented incorporating best practices, known techniques, and patterns that will support the achievement of the business goals. In this way, the complexity can be absorbed, and made tractable.  

Practitioners of complex systems, software engineers, and system designers have been frequently using reference architectures to have a collective understanding of system components, functionalities, data-flows and patterns which shape the overall qualities of system and help further adjust it to the business objectives \cite{Cloutier}; \cite{kohler2019towards}. There is a fair amount of literature on reference architectures, and whereas different authors definition may vary, they all share the same tenets. 

A reference architecture is amalgamation of architectural patterns, standards, software engineering techniques that bridge the problem domain to a class of solutions. This artefact can be partially or completely instantiated and prototyped in a particular business context together with other supporting artefact to enable its use. RAs are often created from previous RAs and architecture \cite{AtaeiACIS}.

The usage of RAs for the development of complex systems is not new. In software product line (SPL) development, RAs are generic artifacts that are configured and instantiated for a particular domain of systems \cite{Derras}. In software engineering, major IT giants like IBM has referred to RAs as the 'best of best practices' to address unique and complex system development challenges \cite{Cloutier}. 

Based on the premises discussed and taking all into consideration, RAs can facilitate the issues of big data architecture and data engineering because of the following reasons;

\begin{enumerate}
    \item RAs can promote adherence to best practice, standards, specifications and patterns
    \item RAs can endow the data architecture team with openness and increase operability, incorporating architectural patterns that ensue desirable predefined quality attributes
    \item RAs can be the best initial start to the big data journey, capturing design issues when they are still cheap
    \item RAs can bring different stakeholders on the same table and help achieve consensus around major technological constructs
    \item RAs can be effective in identifying and addressing cross-cutting concerns
    \item RAs can serve as the organizational memory around design decisions, enlightening next subsequent decisions 
    \item RAs can act as a summary and blueprint in the portfolio of software engineers and architect, resulting in better dissemination of knowledge
\end{enumerate}

\section{Research Methodology}



\section{Front matter}

The author names and affiliations could be formatted in two ways:
\begin{enumerate}[(1)]
\item Group the authors per affiliation.
\item Use footnotes to indicate the affiliations.
\end{enumerate}
See the front matter of this document for examples. You are recommended to conform your choice to the journal you are submitting to.

\section{Bibliography styles}

There are various bibliography styles available. You can select the style of your choice in the preamble of this document. These styles are Elsevier styles based on standard styles like Harvard and Vancouver. Please use Bib\TeX\ to generate your bibliography and include DOIs whenever available.

Here are two sample references: \cite{Feynman1963118,Dirac1953888}.

\section*{References}

\bibliography{mybibfile}

\end{document}